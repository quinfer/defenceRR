% Options for packages loaded elsewhere
\PassOptionsToPackage{unicode}{hyperref}
\PassOptionsToPackage{hyphens}{url}
\PassOptionsToPackage{dvipsnames,svgnames,x11names}{xcolor}
%
\documentclass[
  letterpaper,
  DIV=11,
  numbers=noendperiod]{scrartcl}

\usepackage{amsmath,amssymb}
\usepackage{iftex}
\ifPDFTeX
  \usepackage[T1]{fontenc}
  \usepackage[utf8]{inputenc}
  \usepackage{textcomp} % provide euro and other symbols
\else % if luatex or xetex
  \usepackage{unicode-math}
  \defaultfontfeatures{Scale=MatchLowercase}
  \defaultfontfeatures[\rmfamily]{Ligatures=TeX,Scale=1}
\fi
\usepackage{lmodern}
\ifPDFTeX\else  
    % xetex/luatex font selection
\fi
% Use upquote if available, for straight quotes in verbatim environments
\IfFileExists{upquote.sty}{\usepackage{upquote}}{}
\IfFileExists{microtype.sty}{% use microtype if available
  \usepackage[]{microtype}
  \UseMicrotypeSet[protrusion]{basicmath} % disable protrusion for tt fonts
}{}
\makeatletter
\@ifundefined{KOMAClassName}{% if non-KOMA class
  \IfFileExists{parskip.sty}{%
    \usepackage{parskip}
  }{% else
    \setlength{\parindent}{0pt}
    \setlength{\parskip}{6pt plus 2pt minus 1pt}}
}{% if KOMA class
  \KOMAoptions{parskip=half}}
\makeatother
\usepackage{xcolor}
\setlength{\emergencystretch}{3em} % prevent overfull lines
\setcounter{secnumdepth}{-\maxdimen} % remove section numbering
% Make \paragraph and \subparagraph free-standing
\ifx\paragraph\undefined\else
  \let\oldparagraph\paragraph
  \renewcommand{\paragraph}[1]{\oldparagraph{#1}\mbox{}}
\fi
\ifx\subparagraph\undefined\else
  \let\oldsubparagraph\subparagraph
  \renewcommand{\subparagraph}[1]{\oldsubparagraph{#1}\mbox{}}
\fi


\providecommand{\tightlist}{%
  \setlength{\itemsep}{0pt}\setlength{\parskip}{0pt}}\usepackage{longtable,booktabs,array}
\usepackage{calc} % for calculating minipage widths
% Correct order of tables after \paragraph or \subparagraph
\usepackage{etoolbox}
\makeatletter
\patchcmd\longtable{\par}{\if@noskipsec\mbox{}\fi\par}{}{}
\makeatother
% Allow footnotes in longtable head/foot
\IfFileExists{footnotehyper.sty}{\usepackage{footnotehyper}}{\usepackage{footnote}}
\makesavenoteenv{longtable}
\usepackage{graphicx}
\makeatletter
\def\maxwidth{\ifdim\Gin@nat@width>\linewidth\linewidth\else\Gin@nat@width\fi}
\def\maxheight{\ifdim\Gin@nat@height>\textheight\textheight\else\Gin@nat@height\fi}
\makeatother
% Scale images if necessary, so that they will not overflow the page
% margins by default, and it is still possible to overwrite the defaults
% using explicit options in \includegraphics[width, height, ...]{}
\setkeys{Gin}{width=\maxwidth,height=\maxheight,keepaspectratio}
% Set default figure placement to htbp
\makeatletter
\def\fps@figure{htbp}
\makeatother
\newlength{\cslhangindent}
\setlength{\cslhangindent}{1.5em}
\newlength{\csllabelwidth}
\setlength{\csllabelwidth}{3em}
\newlength{\cslentryspacingunit} % times entry-spacing
\setlength{\cslentryspacingunit}{\parskip}
\newenvironment{CSLReferences}[2] % #1 hanging-ident, #2 entry spacing
 {% don't indent paragraphs
  \setlength{\parindent}{0pt}
  % turn on hanging indent if param 1 is 1
  \ifodd #1
  \let\oldpar\par
  \def\par{\hangindent=\cslhangindent\oldpar}
  \fi
  % set entry spacing
  \setlength{\parskip}{#2\cslentryspacingunit}
 }%
 {}
\usepackage{calc}
\newcommand{\CSLBlock}[1]{#1\hfill\break}
\newcommand{\CSLLeftMargin}[1]{\parbox[t]{\csllabelwidth}{#1}}
\newcommand{\CSLRightInline}[1]{\parbox[t]{\linewidth - \csllabelwidth}{#1}\break}
\newcommand{\CSLIndent}[1]{\hspace{\cslhangindent}#1}

\usepackage{booktabs}
\usepackage{longtable}
\usepackage{array}
\usepackage{multirow}
\usepackage{wrapfig}
\usepackage{float}
\usepackage{colortbl}
\usepackage{pdflscape}
\usepackage{tabu}
\usepackage{threeparttable}
\usepackage{threeparttablex}
\usepackage[normalem]{ulem}
\usepackage{makecell}
\usepackage{xcolor}
\KOMAoption{captions}{tableheading}
\makeatletter
\makeatother
\makeatletter
\makeatother
\makeatletter
\@ifpackageloaded{caption}{}{\usepackage{caption}}
\AtBeginDocument{%
\ifdefined\contentsname
  \renewcommand*\contentsname{Table of contents}
\else
  \newcommand\contentsname{Table of contents}
\fi
\ifdefined\listfigurename
  \renewcommand*\listfigurename{List of Figures}
\else
  \newcommand\listfigurename{List of Figures}
\fi
\ifdefined\listtablename
  \renewcommand*\listtablename{List of Tables}
\else
  \newcommand\listtablename{List of Tables}
\fi
\ifdefined\figurename
  \renewcommand*\figurename{Figure}
\else
  \newcommand\figurename{Figure}
\fi
\ifdefined\tablename
  \renewcommand*\tablename{Table}
\else
  \newcommand\tablename{Table}
\fi
}
\@ifpackageloaded{float}{}{\usepackage{float}}
\floatstyle{ruled}
\@ifundefined{c@chapter}{\newfloat{codelisting}{h}{lop}}{\newfloat{codelisting}{h}{lop}[chapter]}
\floatname{codelisting}{Listing}
\newcommand*\listoflistings{\listof{codelisting}{List of Listings}}
\makeatother
\makeatletter
\@ifpackageloaded{caption}{}{\usepackage{caption}}
\@ifpackageloaded{subcaption}{}{\usepackage{subcaption}}
\makeatother
\makeatletter
\@ifpackageloaded{tcolorbox}{}{\usepackage[skins,breakable]{tcolorbox}}
\makeatother
\makeatletter
\@ifundefined{shadecolor}{\definecolor{shadecolor}{rgb}{.97, .97, .97}}
\makeatother
\makeatletter
\makeatother
\makeatletter
\makeatother
\ifLuaTeX
  \usepackage{selnolig}  % disable illegal ligatures
\fi
\IfFileExists{bookmark.sty}{\usepackage{bookmark}}{\usepackage{hyperref}}
\IfFileExists{xurl.sty}{\usepackage{xurl}}{} % add URL line breaks if available
\urlstyle{same} % disable monospaced font for URLs
\hypersetup{
  pdftitle={Investigating Extreme Linkage Topology in the Aerospace and Defence Industry},
  pdfauthor={Elie Bouri; Barry Quinn; Lisa Sheenan},
  colorlinks=true,
  linkcolor={blue},
  filecolor={Maroon},
  citecolor={Blue},
  urlcolor={Blue},
  pdfcreator={LaTeX via pandoc}}

\title{Investigating Extreme Linkage Topology in the Aerospace and
Defence Industry}
\author{Elie Bouri \and Barry Quinn \and Lisa Sheenan}
\date{}

\begin{document}
\maketitle
\ifdefined\Shaded\renewenvironment{Shaded}{\begin{tcolorbox}[frame hidden, breakable, interior hidden, boxrule=0pt, sharp corners, borderline west={3pt}{0pt}{shadecolor}, enhanced]}{\end{tcolorbox}}\fi

\begin{verbatim}
[1] 72
\end{verbatim}

\hypertarget{abstract}{%
\subsection*{Abstract}\label{abstract}}
\addcontentsline{toc}{subsection}{Abstract}

This paper analyses return and volatility spillovers among 21 global
aerospace and defence (A\&D) companies from six countries and three
continents using quantile-based models and daily data from August 23,
2010, to July 1, 2022. The results show that both return and volatility
spillovers vary over time, and those estimated at normal market
conditions, intensify during COVID-19 and Russia-Ukraine war periods.
Spillovers of returns estimated at lower and upper quantiles exceed
those estimated at the middle quantile. Volatility spillover is
extremely high at the upper quantile and exhibits low variability.
Chinese defence stocks are segmented from the rest under normal return
conditions and a moderate volatility state. In contrast, they are
somewhat integrated under extreme return conditions and volatility
states. Hence, Chinese defence stocks entail more diversification
benefits under normal conditions than in bear or bull markets. Further
analysis shows that geopolitical risk consistently plays a significant
role in driving both returns and volatility spillovers, especially
during the pandemic and war periods, without ignoring the role of
macroeconomic and financial variables. These results have implications
for investors concerned with stock portfolio management under various
return and volatility conditions and for policymakers preoccupied with
policy design under unstable periods

\textbf{keywords:} Aerospace and defence companies; Ukrainian war;
Russia; quantile vector-autoregression; COVID-19.

\hypertarget{introduction}{%
\section{Introduction}\label{introduction}}

On 24 February 2022, the mounting tension between Russia and Ukraine
peaked, instigating a brutal war that has led to wide-scale devastation,
especially in Europe, the consequences of which will be felt far into
the future. While the humanitarian effects are almost incomprehensible,
this destructive war has substantially affected the commodity markets,
notably energy and grain prices, the global economy, the financial
markets, and the fortunes of defence companies. There appears to be no
end to the war, and the spending on defence has experienced a notable
increase globally. In 2022, global military expenditure surpassed
\(US 2.2 trillion for the first time. Large increases in military spending are noticed in Europe in response to the Russia-Ukraine war. To top the chart, the United States (US) spent the most on military spending (\)US
877 billion), followed by China (\(US 292 billion), Russia (\)US 86.0
billion), India (\(US 81.0 billion), Saudi Arabia (\)US 75.0 billion),
the United Kingdom (UK) (\(US 69.0 billion), Germany (\)US 56.0
billion), and France ((\$US 54.0 billion)

On 24 February 2022, Russia attacked Ukraine, initiating a war that has
led to wide scale devastation, especially in Europe, the consequences of
which will be felt far into the future. While the humanitarian effects
are almost incomprehensible, this destructive event has also
substantially affected financial markets, the global economy, energy and
grain prices, and the fortunes of defense companies. There unfortunately
appears to be no end in sight for the war and increases in spending on
defense continued to increase globally. In 2022 global military
expenditure surpassed \$US 2 trillion for the first time
(\href{https://www.sipri.org/media/press-release/2022/world-military-expenditure-passes-2-trillion-first-time}{World
military expenditure passes \$2 trillion for first time according to
SIPRI}). During the same year, the United States (US) spent the most on
military spending (\$US 750 billion), followed by China (\$US 237
billion), Saudi Arabia (\$US 67.6 billion), India (\$US 61 billion) and
the United Kingdom (UK) (\$US 55.1 billion).\footnote{Data are according
  to Stockholm International Peace Research Institute
  (\url{https://sipri.org/})}

This defence spending has seen the global aerospace and defence (A\&D)
industry outperforms the equity markets. According to Refinitiv Eikon,
for the year ending 31/12/2022, the total return of the A\&D industry
equated to 13.89\%, compared to -3.45\% for the global equity markets.
This is even though the average total market capitalisation of A\&D
companies represents only 1\% of the global equity markets. Total
revenues of the year ending 31/12/2022 were \$US 665 billion\footnote{All
  market analysis was conducted using Refinitiv Eikon on 22/02/2023. The
  A\&D sample consisted of 351 publicly listed securities with a total
  market capitalisation of \$1.37 trillion dollars. The global equity
  market capitalisation at this time was 118 trillion dollars.}, with
the top 21 largest companies by revenue capturing 75\% of this total
revenue (See Appendix Table A1 for detailed Market Analysis). Merger and
acquisition activity in the A\&D industry was also high, with 361 deals
with a total value, including Net Debt, of \$US 36 billion. Given the
Russia and Ukraine war and the outsize recent market performance of many
A\&D companies, our study investigates the network topology of the
return and volatility spillovers among major companies from the global
A\&D industry and the factors that drive these spillovers. Using a
quantile-based approach to spillovers, we explicitly consider the
transmission pathway of return and volatility from one company to
another under various market return conditions and volatility states.
Furthermore, we conduct various regressions to understand the role of
geopolitical risk in driving return and volatility spillovers while
considering various macroeconomic and financial variables.

Surprisingly, the related literature could be more extensive in this
regard. Studies on the A\&D industry date back at least as far as the
1960s, mainly focusing on the investment quality of companies in this
industry (Butler Jr, 1966b, 1966a, 1967) and their profits and market
performance (Agapos and Gallaway, 1970; Suarez, 1976; Bohi, 1973).
McDonald and Kendall (2011) studied the effects of war on the U.S.
defence industry, focusing on 16 firms that provided military equipment
to the Department of Defence. Applying a cumulative prediction error
(CPE) technique, they find that stock prices of defence firms tend to
increase because of military actions. Capelle-Blancard and Couderc
(2008) analyse the effect of media information on defence companies
only, showing that news relating to earnings announcements and analyst
recommendations are significant in explaining abnormal returns for these
companies. Recently, Federle et al.~(2022) analysed stock market
responses to the war in Ukraine, finding that firms closer to Ukraine
suffered from a relative proximity penalty, experiencing negative equity
returns during the four weeks surrounding the beginning of the war. Le
et al.~(2023) use war-related news articles to investigate the market
response of some companies to the war in Ukraine, showing a negative
impact on airline stocks and a positive impact on defence stocks. Zhang
et al.~(2022) consider the co-movements between geopolitical risk and
the returns and volatility of global aerospace and defence companies,
indicating significant co-movements around the onset of the war in
Ukraine. This is labelled a `flight-to-arms' phenomenon, with
co-movement found to be significant for many European and US companies
in the sample. This paper contributes to the above body of literature by
analysing the network of returns and volatility spillovers across major
A\&D companies, using a quantile-based connectedness, and covering the
drivers of spillovers. First, the flexibility of this connectedness
approach allows for accounting for various market return conditions and
volatility states. This is an important feature as previous findings
highlight the importance of considering the sign of return shocks and
size of volatility shocks when studying the spillover effects in the
financial markets (see, Bouri et al., 2020; Chatziantoniou et al., 2021;
Saeed et al., 2021; Iqbal et al., 2022). This should add to Zhang et
al.~(2022) and Le et al.~(2023), who consider only mean-based models.
Second, our sample of 21 A\&D companies incorporates eight out of the
ten largest in the world by revenue and covers six countries, namely US,
UK, France, Germany, China and Singapore, across three continents (North
America, Europe, and Asia) over the period August 23, 2010 to July 1,
2022. The period under study covers important events such as the Russian
invasion of Crimea in 2014, the COVID-19 pandemic of 2020, and the
Russia-Ukraine in 2022, thus enabling the identification of significant
spillovers across a varying set of global turbulent market conditions
and geopolitical events. Third, analysing the factors driving the return
and volatility spillovers across various market conditions constitutes
another contribution to the literature on A\&D companies (Federle et
al., 2022; Le et al., 2023; Zhang et al., 2022). Specifically, it shows
the importance of geopolitical risk in driving up the spillover effect
of both returns and volatility while accounting for the significance of
macroeconomic and financial variables. The main results on the spillover
effects across various quantiles show intensified spillover effects for
both stock returns and volatility under extreme market conditions and
evidence of time evolution in the spillovers, especially for lower and
upper tails returns. This suggests that the system of spillovers under
extreme market conditions, irrespective of whether it is bull or bear
market, is unstable. Thus, it should be carefully monitored, given its
potential consequences on portfolio and risk management in the A\&D
industry. An additional analysis involving the drivers of spillovers
shows the impact of heightened geopolitical risk around the
Russian-Ukraine war period on the return and volatility spillovers
across A\&D companies in most of the quantiles considered. Furthermore,
the level of returns and volatility spillovers is also driven by
macroeconomic and financial variables such as corporate credit
conditions, stock market volatility, short-term liquidity risk, and real
business condition. Therefore, participants in the A\&D industry,
especially in the US and Europe, should closely examine the global
geopolitical environment given its significant impact on the dynamics of
information transmission in the A\&D industry and, thus, the integration
of A\&D stocks and the possibilities of diversification. The findings
also highlight the tail risk propagation within the network system of
aerospace and defence stocks, which should concern risk managers and
policymakers. The paper proceeds as follows. Section 2 describes the
dataset. Section 3 provides a quantile-based spillover approach. Section
4 presents and discusses the results. Section 5 concludes.

\hypertarget{data}{%
\section{Data}\label{data}}

Our dataset comprises the daily closing prices of 21 global aerospace
and defence companies in six countries (US, UK, France, Germany, China,
Singapore) and three continents (North America, Europe, and Asia). The
selected companies are chosen to be large and liquid, with an individual
market capitalisation exceeding e billion USD. The list of 21 companies
is provided in Appendix Table A1. Appendix Figure A1 shows the share our
sample captures of the total market capitalisation of the global A\&D
industry, which represents 72 \%. The sample period (August 23, 2010 -
July 1, 2022) is selected according to the availability of A\&D stock
price data from Refinitiv DataStream, especially for the Chinese Aecc
Aviation Power `A' because it exhibited many zero price fluctuations at
the daily basis before the start of the sample period on 23 August 2010.

Appendix Figure A2 plots the price series levels and highlights the
country of incorporation. Furthermore, Appendix Figure A3 and Figure A4
display the log-return and volatility series, respectively. The price
series levels reveal several distinct groupings in their movements. For
many cases, price series levels in Figure A2 show common movement with a
regime shift towards higher prices and larger volatility around 2020.
Notably, Chinese stocks experienced a shock in 2016 of a similar
magnitude to that of 2020. In July 2016, it was reported that China had
performed a week of military drills in the South China Sea amid legal
debates regarding its territorial claims to regional areas, which could
account for this increase in volatility. In Figure A3, we notice a large
variability in the returns around the peak of COVID-19 in 2020 and the
Russia-Ukraine war in early 2022, especially for most US-European A\&D
stocks. Based on Figure~A4, we observe that the volatility of some A\&D
companies such as Boeing, Transdigm, Safran, and Rolls-Royce Holdings
experienced a spike around the pandemic outbreak and the Russia-Ukraine
war.

Appendix A2 and Table A3 present summary statistics for daily returns
and volatility series, respectively. Notably, Table A2 shows that the
distributions of the daily returns series are mostly skewed to the left
and exhibit fat ``tailedness'', with Airbus, Boeing, Rolls-Royce
Holdings, Safran, and Transdigm experiencing the highest daily standard
deviation. These companies are either directly in the aviation industry
or supply to it, as Rolls-Royce supplies the Trent engine to Airbus. In
this regard, airlines were hit particularly hard during the COVID-19
pandemic, with an estimated economic loss of US\$168 billion in 2020
(COVID-19's impact on the global aviation sector \textbar{}
McKinsey)\emph{LISA add ref please}, which may be a factor in the
observed volatility. All return series are stationary, as shown by the
Augmented Dickey-Fuller (ADF) and Phillips-Perron (PP) test statistics.
The same is true for the volatility series, as reported in Appendix
Table A3.

\hypertarget{tbl-Xtremes}{}
\begin{table}[H]
\caption{\label{tbl-Xtremes}Extremely votality events }\tabularnewline

\centering
\begin{tabular}[t]{llrrl}
\toprule
Date & stock & return & volatility & country\\
\midrule
2020-03-18 & Airbus & -0.25 & 0.06 & France\\
2020-03-16 & Boeing & -0.27 & 0.07 & US\\
2020-11-09 & Rolls-Royce Holdings & 0.36 & 0.13 & UK\\
2020-03-18 & Safran & -0.26 & 0.07 & France\\
2020-03-18 & Transdigm Group & -0.25 & 0.06 & US\\
\bottomrule
\end{tabular}
\end{table}

Furthermore, we consider extreme volatility events, which coincide with
the peak of the COVID-19 outbreak and indicate in Table 1 that the most
volatile A\&D stock is Rolls Royce Holdings. The company reported a loss
of £4 billion for 2020 and was forced to raise £7.3 billion in debt and
equity and cut almost one-fifth of its workforce (COVID-19: Rolls-Royce
blames the `severe impact' of the pandemic as it dives to £4bn loss
\textbar{} Business News \textbar{} Sky News)\emph{LISA add ref please}.
Unsurprisingly, Table~\ref{tbl-Xtremes} indicates that the five highest
daily volatility scores mainly occur around the end of March 2020 at the
height of the uncertainty during the onset of the COVID-19 pandemic.

\hypertarget{methodology}{%
\subsection{Methodology}\label{methodology}}

The nature and strength of spillovers across financial markets have
traditionally been measured using conventional mean estimators such as
the connectedness approach of Diebold and Yılmaz (2014). Interestingly,
Ando et al.~(2022) argue that systemic shocks are likely to be much
larger than average shocks and that extreme negative return (or low
volatility) shocks do not necessarily propagate in the same way as
extreme positive return (or high volatility) shocks. Therefore,
quantile-based estimators allow for identifying whether the spillover
effects' topology changes with the shock distribution's size and sign as
captured by the various quantiles of the shock distribution (Bouri et
al., 2020).

To study the return and volatility connectedness across 21 global
aerospace and defence companies, we use the quantile-VAR-based
connectedness approach following Ando et al.~(2022)\footnote{The
  methodology has been used by Bouri et al.~(2020), Chatziantoniou et
  al.~(2021) and Saeed et al.~(2021).}. This approach extends the
mean-based connectedness approach of Diebold and Yılmaz (2014) and thus
allows for capturing the extreme connectedness estimated at the lower,
middle, and upper quantiles. For returns, this helps obtain the
connectedness of return shocks in bear, normal, and bull markets. For
volatility, it helps capture the connectedness of volatility shocks in
low, middle, and high volatility states.

We consider a portfolio enivroment, where stocks are indexed
i=1,2,\ldots,N, and time periods are indexed t=1,2,\ldots,T. Based on a
quantile regression (Koenker, 2005), we consider a quantile-VAR process
of p\textsuperscript{th} order for a set of N return (volatility) series
for time T, \(y_{it}=\{y_{t=1,i=1},\dots,y_{t=T,i=N}\}\) , as given by:

\[
y_{t}=c_{i(\tau)}+\sum_{j=1}^{p} B_{j,(\tau)} y_{t-j}+e_{t(\tau)}, t=1,\dots,T
\]

where, \(c_{(\tau)}\) denotes a vector of constant terms at quantile τ,
\(B_{j(\tau)}\) represents the matrix of the j\textsuperscript{th}
lagged coefficients of the dependent variable at quantile τ, with i
=1,\ldots, p, and \(e_{t(\tau)}\) denotes a vector of error terms at
quantile τ. Equation (1) is estimated by assuming that the error terms
conform to the population quantile restriction,
\(Q_t(e_{t(\tau)} |y_{t=1},\dots,y_{t=p})=0\) .

We express the τth conditional quantile of response y as:

\[
Q_t(y_t |y_{t=1},\dots,y_{t=p})=c_{(\tau)}+\hat{B_{i(\tau)}} y_{t-i}
\]

Following the approach of Diebold and Yılmaz (2014) , we compute return
and volatility connectedness measures based on a quantile variance
decomposition.

We represent Equation
\href{https://www.sciencedirect.com/science/article/pii/S1062940819304085\#e0015}{(3)}
as an infinite order vector moving average process:

\[
y_t=\mu_{(\tau)}+\sum_{s=0}^{\infty}A_{s(\tau)}e_{t-s(\tau)}, t=1,\dots.T
\]

where,

\begin{align*}
\mu{(\tau)}= \frac{c_{\tau}}{\left (I_n-B_{1(\tau)}-\dots-B_{p(\tau)} \right)} \\
A_{s(\tau)}= \begin{cases} 0, s<0 \\ I_n, s=0 \\ B_{1(\tau)}A_{s-1(\tau)}+\dots+B_{p(\tau)}A_{s-p(\tau)}, s>0 \end{cases} \\
\text{and $y_t$ is given by the sum of $e_{t(\tau)}$}
\end{align*}

The generalized forecast error variance decomposition
(GFEVD),\(\theta^h_{i,j}\), is computed as in Diebold and Yılmaz (2014).
The GFEVD reflects the contribution of the i\textsuperscript{th} stock
return (volatility) to the variance of the forecast error of the stock
return (volatility) i\textsuperscript{th} at h-steps ahead and is
defined as:

\[
\theta^{(h)}_{j \leftarrow i,(\tau)}= \frac{\sigma_{ii}^{-1}\sum_{l=0}^{h}(e_j^{'}h_h \Omega_{(\tau)} e_j)^2}{\sum_{h=0}^{H-1}(e_i^{'}h_h \Omega_{(\tau)} e_i)}
\]

where, V is the variance matrix of the vector of residuals,
\(\sigma_{ii}\) is the j\textsuperscript{th} diagonal element of the V
matrix, and \(e_i\)denotes a vector with a value of 1 for the
i\textsuperscript{th} element and 0 otherwise.

Its scaled version,\(\theta_{j\leftarrow i,(\tau)}^h\) , is represented
as:

\[
\theta_{j\leftarrow i,(\tau)}^h=\frac{\theta^{(h)}_{j \leftarrow i,(\tau)}}{\sum_{j=1}^N \theta^{(h)}_{j \leftarrow i,(\tau)}}
\]

The scaled version measures the spillover of the idiosyncratic shock
affecting variable i onto variable j (Ando, Greenwood-Nimmo, and Shin
2022).

Various spillover measures are estimated at each quantile and are
summarised in Table 2. The third column, ``Description'', describes how
these can be interpreted in terms of the system of spillovers. Note
that, by construction, own share and FROM sum to one for i=1,2,..,m;
however, TO can take values bigger than or less than one.

The lag order of the quantile VARs is selected based on SIC. It equals 1
for the quantile-VAR of the return series and 2 for the quantile-VAR of
the volatility series. As for the forecast horizon (H), we use ten days.
Furthermore, we conduct a time-varying spillover analysis (Diebold \&
Yilmaz, 2014) based on a rolling window of 200 days. To assess the
robustness of our results, we use a fixed window length of 200 days and
a 5-step forecast horizon and show that our spillover results remain
almost the same, suggesting their robustness to the window size and
forecast horizon. These results are not reported here but are available
on request from the authors.

\begin{longtable}[]{@{}
  >{\raggedright\arraybackslash}p{(\columnwidth - 4\tabcolsep) * \real{0.2361}}
  >{\raggedright\arraybackslash}p{(\columnwidth - 4\tabcolsep) * \real{0.2917}}
  >{\raggedright\arraybackslash}p{(\columnwidth - 4\tabcolsep) * \real{0.4722}}@{}}
\caption{Table 2: Description of modelling outputs}\tabularnewline
\toprule\noalign{}
\begin{minipage}[b]{\linewidth}\raggedright
Name
\end{minipage} & \begin{minipage}[b]{\linewidth}\raggedright
Formula
\end{minipage} & \begin{minipage}[b]{\linewidth}\raggedright
Description
\end{minipage} \\
\midrule\noalign{}
\endfirsthead
\toprule\noalign{}
\begin{minipage}[b]{\linewidth}\raggedright
Name
\end{minipage} & \begin{minipage}[b]{\linewidth}\raggedright
Formula
\end{minipage} & \begin{minipage}[b]{\linewidth}\raggedright
Description
\end{minipage} \\
\midrule\noalign{}
\endhead
\bottomrule\noalign{}
\endlastfoot
Own share &
\(                                                                                                                                                                                                                                                                                                                                                                                                                                                                                                                                                                                                                                                                                                                                                                                                                                                                                                                                                                                                                                                                                                                                                                                                                                                                                                                                        
                                                                                                                                                                                                                                                                                                                                                                                                                                                                                                                                                                                                                                                                                                                                                                                                                                                                                                                                                                                                                                                                                                              \tilde{\theta_{j\leftarrow i,(\tau)}^h}                                                                                                                                                                                                   
                                                                                                                                                                                                                                                                                                                                                                                                                                                                                                                                                                                                                                                                                                                                                                                                                                                                                                                                                                                                                                                                                                              \)
& The proportion of the h-steps-ahead GFECD of the ith variable that can
be attributed to the shocks to variable i \\
FROM &
\(                                                                                                                                                                                                                                                                                                                                                                                                                                                                                                                                                                                                                                                                                                                                                                                                                                                                                                                                                                                                                                                                                                                                                                                                                                                                                                                                        
                                                                                                                                                                                                                                                                                                                                                                                                                                                                                                                                                                                                                                                                                                                                                                                                                                                                                                                                                                                                                                                                                                              F_{i \leftarrow \cdot,(\tau)}^h =\sum_{j=1,i \ne j}^m \theta_{j\leftarrow i,(\tau)}^h                                                                                                                                                     
                                                                                                                                                                                                                                                                                                                                                                                                                                                                                                                                                                                                                                                                                                                                                                                                                                                                                                                                                                                                                                                                                                              \)
& Measures the total spillover from the system to i, capturing external
condition effects on i. \\
TO &
\(                                                                                                                                                                                                                                                                                                                                                                                                                                                                                                                                                                                                                                                                                                                                                                                                                                                                                                                                                                                                                                                                                                                                                                                                                                                                                                                                        
                                                                                                                                                                                                                                                                                                                                                                                                                                                                                                                                                                                                                                                                                                                                                                                                                                                                                                                                                                                                                                                                                                              T_{\cdot \leftarrow i,(\tau)}^h =\sum_{j=1,i \ne j}^m \theta_{j\leftarrow i,(\tau)}^h                                                                                                                                                     
                                                                                                                                                                                                                                                                                                                                                                                                                                                                                                                                                                                                                                                                                                                                                                                                                                                                                                                                                                                                                                                                                                              \)
& Measures the total spillover from i to the system, capturing the
influence of ith node in the network. \\
NET &
\(                                                                                                                                                                                                                                                                                                                                                                                                                                                                                                                                                                                                                                                                                                                                                                                                                                                                                                                                                                                                                                                                                                                                                                                                                                                                                                                                        
                                                                                                                                                                                                                                                                                                                                                                                                                                                                                                                                                                                                                                                                                                                                                                                                                                                                                                                                                                                                                                                                                                              T_{\cdot \leftarrow i,(\tau)}^h -F_{i \leftarrow \cdot,(\tau)}^h                                                                                                                                                                          
                                                                                                                                                                                                                                                                                                                                                                                                                                                                                                                                                                                                                                                                                                                                                                                                                                                                                                                                                                                                                                                                                                              \)
& Meaures the directional connectedness of variable i. \\
TOTAL &
\(                                                                                                                                                                                                                                                                                                                                                                                                                                                                                                                                                                                                                                                                                                                                                                                                                                                                                                                                                                                                                                                                                                                                                                                                                                                                                                                                        
                                                                                                                                                                                                                                                                                                                                                                                                                                                                                                                                                                                                                                                                                                                                                                                                                                                                                                                                                                                                                                                                                                              S_{\tau}^h=m^{-1}\sum                                                                                                                                                                                                                     
                                                                                                                                                                                                                                                                                                                                                                                                                                                                                                                                                                                                                                                                                                                                                                                                                                                                                                                                                                                                                                                                                                              F_{i \leftarrow \cdot,(\tau)}^h                                                                                                                                                                                                           
                                                                                                                                                                                                                                                                                                                                                                                                                                                                                                                                                                                                                                                                                                                                                                                                                                                                                                                                                                                                                                                                                                              \)
& Is the sum of the from system estimates. \\
\end{longtable}

\hypertarget{results}{%
\section{Results}\label{results}}

In the context of global defence stocks, we are primarily interested in
the spillover effects due to crisis periods and conflict events,
particularly relevant in the current geopolitical climate. Regarding
financial risk management, the propagation of idiosyncratic risk
contagion is often defined as the difference in how the shock propagates
during extreme events relative to normal times (Londono, 2019). Our
analysis thus attempts to investigate how much of the uncertainty
associated with A\&D stocks can be attributed to the idiosyncratic
shocks coming from these stocks as the shock size varies.

We present the return and volatility spillovers across the 21 A\&D
stocks under study. The sample period of August 23, 2010 - July 1, 2022,
covers normal and extreme market conditions, including the COVID-19
outbreak and the Russia-Ukraine war.

\hypertarget{network-topology-of-return-and-volatility-spillovers}{%
\subsection{Network topology of return and volatility
spillovers}\label{network-topology-of-return-and-volatility-spillovers}}

To understand the aggregate spillover intensity among A\&D stocks, we
visualise the results of a full-sample analysis for both returns and
volatility at the median, 5th and 95th percentile. The network
visualisation reflects the strength of the bilateral spillovers by the
relative thickness of the edges. At the same time, the size of each node
is proportional to the square root of the total spillover (inwards and
outwards) (Ando et al., 2022). Finally, the country of origin of the
company is represented by colour.

\hypertarget{return-spillovers}{%
\subsubsection{Return spillovers}\label{return-spillovers}}

Figure~\ref{fig-rtn} illustrates the network visualisation of the
bilateral spillover effects of the 21 return series, whereas Figure 6
displays the exact visualisation for the 21 volatility series. Some
similar patterns emerge, notably the consistent size of the US stock
nodes representing the significant aggregate spillover effects in both
directions. In all plots, Raytheon Technology experiences the most
considerable aggregate spillover effects, indicative of its dominance in
the A\&D industry. However, there are also some crucial differences.
Firstly, the most substantial individual pairwise spillover effects are
observed at the median conditional distribution, primarily within
countries. Notably, Chinese stocks show the most robust linkages within
the country but the weakest linkages outside their country of origin.
This corresponds to literature relating to general stock market trends
observed in China; for example, Valukonis (2014) finds that following
the recovery from the financial crisis of 2008, Chinese and US stock
market indices display a weak correlation which is perhaps due to
Chinese markets being somewhat isolated from global markets and not as
influenced by globalisation as other markets may be.

In contrast, all pairwise spillover effects are weaker at the extremes
of the return distribution. This finding is consistent with previous
studies, which show that in times of stress, the network is
characterised by a more significant number of weaker bilateral linkages
increasing the weight completeness of the network (Dungey et al., 2019;
Ando et al., 2022). In our context, this would mean that while shock
spillovers between individual stock pairs are small, the overall
connectedness of the system is increasing in times of stress, meaning
that the shock propagation is higher under extreme return conditions (
).

\begin{figure}

{\centering 

\begin{figure}[H]

{\centering \includegraphics[width=6.75in,height=\textheight]{plots/fig-rtn50.png}

}

\caption{Middle quantile (50th Percentile)}

\end{figure}

\begin{figure}[H]

{\centering \includegraphics[width=6.75in,height=\textheight]{plots/fig-rtn95.png}

}

\caption{Extreme upper quantile (95th percentile)}

\end{figure}

\begin{figure}[H]

{\centering \includegraphics[width=6.75in,height=\textheight]{plots/fig-rtn5.png}

}

\caption{Extreme lower quantile (5th percentile)}

\end{figure}

}

\caption{\label{fig-rtn}Network topology of return spillover at various
quantiles}

\end{figure}

\hypertarget{volatility-spillovers}{%
\subsubsection{Volatility spillovers}\label{volatility-spillovers}}

Moving to the network of volatility spillovers, Figure 6 exhibits a
similar pattern to those for the return spillovers. Weak bilateral
spillovers characterise the extreme upper quantile, and the strongest
pairwise spillovers occur at the median quantile. Again, we observe the
most robust volatility linkages between Chinese companies, followed by
volatility linkages between US companies. Strong volatility linkages
between Chinese companies are also apparent at the extreme lower
quantile. Huang et al.~(2021) constructed a tail risk spillover network
for China's industry sectors. They showed that the national defence
sector is defined as a `downstream' sector due to its position in the
industrial chain. It is found that it has relatively high volatility
compared to other leading industries. Bu et al.~(2019) analysed movement
in the Chinese stock market using a causal network method, finding that
investors are concerned with risk and return in normal periods but are
only concerned about risk in crisis periods.

\begin{figure}

{\centering 

\begin{figure}[H]

{\centering \includegraphics[width=6.75in,height=\textheight]{plots/fig-vol50.png}

}

\caption{Middle quantile (50th Percentile)}

\end{figure}

\begin{figure}[H]

{\centering \includegraphics[width=6.75in,height=\textheight]{plots/fig-vol95.png}

}

\caption{Extreme upper quantile (95th percentile)}

\end{figure}

\begin{figure}[H]

{\centering \includegraphics[width=6.75in,height=\textheight]{plots/fig-vol5.png}

}

\caption{Extreme lower quantile (5th percentile)}

\end{figure}

}

\caption{\label{fig-vol}Network topology of volatility spillover at
various quantiles}

\end{figure}

\hypertarget{time-varying-spillover-results}{%
\subsection{Time varying spillover
results}\label{time-varying-spillover-results}}

So far, we have analysed measures of connectedness for the entire sample
using the network topology visualisation. However, it is essential to
illustrate meaningful time variation in the returns and volatility
spillover effects of A\&D stocks under various market conditions.
Furthermore, bilateral spillover of idiosyncratic risk seems stronger
for both returns and volatility series, which reflect the
interconnectedness across A\&D stocks. However, it is important to note
that restricting the network analysis to the middle of the distribution
will not capture the full extent of dependence when large negative
return and large positive return shocks occur (i.e.~under extreme market
conditions and events) as well as very low and very high volatility
shocks manifest. Therefore, in this section, we conduct a rolling
analysis with a quantile VAR to capture the time variability in the
return and volatility spillovers in normal times (i.e.~at the median of
the conditional distribution) and abnormal market conditions
(i.e.~at~the upper and lower tails of the conditional distribution). We
use a fixed window length of 200\footnote{Existing studies in the
  Deibold-Yilmaz network literature use windows ranging from 100-250
  days. Sensitivity analysis has been done and available upon request.}
days and a 10-step forecast horizon. This will provide a comprehensive
analysis of connectedness at the center and in the left and right tail
dependence. This is conducted for both returns and volatility.

\hypertarget{total-system-connectivity}{%
\subsubsection{Total system
connectivity}\label{total-system-connectivity}}

\begin{figure}[H]

{\centering \includegraphics{plots/fig-TCI50.png}

}

\caption{\label{fig-TCI50}Conditional Median}

\end{figure}

\begin{figure}[H]

{\centering \includegraphics{plots/fig-TCI5.png}

}

\caption{\label{fig-TCI5}5th percentile}

\end{figure}

\begin{figure}[H]

{\centering \includegraphics{plots/fig-TCI95.png}

}

\caption{\label{fig-TCI95}95\textasciitilde th\textasciitilde{}
percentile}

\end{figure}

\begin{figure}[H]

{\centering \includegraphics{plots/fig-TCIrtd.png}

}

\caption{\label{fig-TCIrtd}95th minus 5th percentile}

\end{figure}

Figure~\ref{fig-TCI50}, Figure~\ref{fig-TCI5}, Figure~\ref{fig-TCI95},
comparing total connectedness of returns and volatility. Vertical red
dashed lines denote the dates of some important events, which are
described in Table~\ref{tbl-dates}.

\hypertarget{tbl-dates}{}
\begin{table}[H]
\caption{\label{tbl-dates}Important Dates }\tabularnewline

\centering
\begin{tabular}[t]{lll}
\toprule
label & date & description\\
\midrule
a & 2014-02-20 & Russia began annexation of Crimea\\
b & 2014-04-07 & Start of war in Donbas by pro-Russian activists\\
c & 2014-10-15 & October 2014 flash crash\\
d & 2016-06-23 & Brexit referendum\\
e & 2016-12-14 & Federal Reserve raises interest rates\\
\addlinespace
f & 2017-03-29 & the United Kingdom invokes article 50 of the Lisbon Treaty\\
g & 2017-06-08 & snap election held in the United Kingdom\\
h & 2020-03-18 & Dash for cash crisis in bond market peaks\\
i & 2022-02-24 & Russia initiated a special military operation in Donbas\\
\bottomrule
\end{tabular}
\end{table}

The TOTAL connectedness index at the conditional median (a measure of
the average connectedness) and extremes for returns and volatility
systems are presented in Figure~\ref{fig-TCI50}. In normal conditions
the connectedness in the returns system tends to be larger than that of
the volatility system of defense stocks. The connectedness reaches its
peak at point h (the `dash for cash' event) at the beginning of the
COVID-19 pandemic. Importantly, while the connectedness levels are
greater in the returns system the volatility system connectedness
exhibits higher sensitivity to shocks, with the largest regime shift at
point h.

Figure~\ref{fig-TCI5} and Figure~\ref{fig-TCI95} illustrate the time
variation of total system connectivity at the 5th and 95th percentiles
of the conditional distributions. It is noted that return system
connectedness is persistently high (above 90) at both tails of the
conditional distribution, while volatility system connectedness in
period of extremely low volatility (5th percentile) is more sensitivity
temporal events.

In the spirit of Ando et al.~(2022), we illustrate in
Figure~\ref{fig-TCIrtd} the relative tail dependence (RTD) calculated as
the difference between the 95th and 5th percentile for both returns and
volatility spillovers. Positive (negative) values of RTD indicate
stronger (weaker) dependence in the right tail compared to the left
tail. For returns, we interpret increases (decreases) in RTD as evidence
of a rising (falling) connectedness of the financial performance of
defence stocks. For volatility, we interpret increases (decreases) in
RTD as evidence of rising (falling) connectedness of financial
uncertainty in defence stocks, or more succinctly, rising (falling)
financial fragility as positive (negative) volatility shocks disseminate
through the system of defence stocks.

Starting with the illustration of the RTD of the return spillovers, the
upper panel of Figure~\ref{fig-TCIrtd} shows a time variation of the RTD
for return spillovers and evidence of asymmetric effect over the period,
indicative of non-equally spread of positive and negative feedback loops
in return spillover effects. Moving to the lower panel of
Figure~\ref{fig-TCIrtd}, we notice the persistent one-sidedness of the
RTD for the volatility series, with the right tail of the condition
distribution dominant throughout the period. This asymmetry suggests
that the size of that uncertainty amplifies volatility spillovers across
the defence. These results suggest that the total connectedness across
the shocks of A\&D stocks is affected by the sign of returns and the
size of the volatility in the system.

Furthermore, we consider the chronological order of prominent global
economic and conflict turmoil events in the context of median and
extreme spillovers in volatility and return shocks. Some striking
patterns emerge in this chronological order. From the beginning of the
conflict in Crimea (a + b) to the Brexit referendum (d), the RTD for
volatility trends down, which is indicative of an increase in resilience
(reduction in fragility) in the system of A\&D stocks. This is coupled
with the fact that RTD is primarily positive for the return system in
this sub-period. This finding suggests that upper tail returns (right
tail of the conditional distribution) in this period induce some
spillover effects while the financial fragility of the system weakens.
There is also a notable regime shift at the dash for cash date (h),
where the financial fragility (the volatility system) fell by 50\% (TCI
= 40 to TCI = 20).

\hypertarget{individual-connectivity}{%
\subsubsection{Individual connectivity}\label{individual-connectivity}}

To disentangle the total connectedness variation further explore the net
spillover effects
\(T_{\cdot \leftarrow i,(\tau)}^h -F_{i \leftarrow \cdot,(\tau)}^h\).
Figure~\ref{fig-netrtn} presents the Net spillover of individual stock
returns at the median, lower and upper quantiles.
Figure~\ref{fig-netvol} presents the same estimates for stock
volatilities.

\begin{figure}

{\centering 

\begin{figure}[H]

{\centering \includegraphics[width=6.75in,height=\textheight]{plots/fig-rtnnet50.png}

}

\caption{Middle quantile (50th Percentile)}

\end{figure}

\begin{figure}[H]

{\centering \includegraphics[width=6.75in,height=\textheight]{plots/fig-rtnnet95.png}

}

\caption{Extreme upper quantile (95th percentile)}

\end{figure}

\begin{figure}[H]

{\centering \includegraphics[width=6.75in,height=\textheight]{plots/fig-rtnnet5.png}

}

\caption{Extreme lower quantile (5th percentile)}

\end{figure}

}

\caption{\label{fig-netrtn}Net spillover effects for the stock returns
at various quantiles}

\end{figure}

We group these plots by country, and some interesting patterns emerge.
Firstly, at the median of distributions, the three Chinese defence
stocks are net spillover receivers in both their return performance and
volatility. This may indicate the need for global maturity in these
stocks compared to the other members of the system belonging to
developed markets (e.g., the US and Europe). Secondly, the US A\&D
stocks dominate the sample and are net transmitters of both volatility
and return spillover effects. More precisely, in normal periods
(i.e.~the median of the conditional distribution), Raytheon Technologies
and General Dynamics are dominant net transmitters. This pattern also
replicates at the extremes of the conditional distributions. While this
is unsurprising given that Raytheon Technologies is the largest global
defence stock, it is worth noting that General Dynamics is the sixth
largest. For the latter, the result can be driven by some sizeable
recent defence contracts signed, for example, the US National
Geospatial-Intelligence Agency in March 2022 (US\$4.5 billion), the US
Navy in August 2022 (US\$1.4 billion) and the US Army in 2022 (US\$1.2
billion). Compared to the system of returns, the system of volatilities
exhibits much more time variation, perhaps indicative of the high
sensitivity to market fluctuations of financial risk.

\begin{figure}

{\centering 

\begin{figure}[H]

{\centering \includegraphics[width=6.75in,height=\textheight]{plots/fig-volnet50.png}

}

\caption{Middle quantile (50th Percentile)}

\end{figure}

\begin{figure}[H]

{\centering \includegraphics[width=6.75in,height=\textheight]{plots/fig-volnet95.png}

}

\caption{Extreme upper quantile (95th percentile)}

\end{figure}

\begin{figure}[H]

{\centering \includegraphics[width=6.75in,height=\textheight]{plots/fig-volnet5.png}

}

\caption{Extreme lower quantile (5th percentile)}

\end{figure}

}

\caption{\label{fig-netvol}Net spillover effects for the stock
volatilities at various quantiles}

\end{figure}

In terms of the prominent dates in Figure~\ref{fig-netvol}, there are
notable positive spikes at the start of the COVID-19 pandemic with the
largest appearing in the median of the conditional distribution of the
volatility system. The largest of these are in Raytheon Technologies and
Howmet Aerospace, which both spike at over 200 in net transmission
terms. Within the US stocks, Raytheon and General Dynamics are the most
transmitive in both their median and extreme spillover effects. Finally,
in terms of magnitude, Singapore technology engineering, are the largest
receiver of spillover effects at both the median and the extremes, which
is not surprising given their small market capitalization compared to
the others (see Table A1 for details of size of stocks).

Compared to previous studies, our above findings reveal that both return
and volatility spillovers are unstable over time, and those estimated at
normal market conditions (at the middle quantile), intensify during
crisis periods such as the COVID-19 outbreak. There is also evidence of
intensified spillover effects for return shocks at both lower and upper
quantiles, exceeding the return spillover effects estimated at the
middle quantile, thus indicating significantly different behaviour of
spillovers across different market conditions. The level of spillovers
at the lower quantile in the return system is considerably larger than
that in the volatility system. However, the level of volatility
spillover is exceptionally high at the upper quantile only and exhibits
a low variability. Finally, Chinese defence stocks seem segmented from
the rest under normal return conditions and a moderate volatility state.
However, they somewhat integrated with global defence stocks under
extreme return conditions and volatility states. This implies that
Chinese defence stocks entail more diversification benefits under normal
conditions than bear or bull markets when returns and volatility are
very low or high.

\hypertarget{drivers-of-return-and-spillovers-the-role-of-geopolitical-risk}{%
\subsection{Drivers of return and spillovers -- the role of geopolitical
risk}\label{drivers-of-return-and-spillovers-the-role-of-geopolitical-risk}}

In this section, we provide insights into the main drivers of return and
volatility spillovers across A\&D stocks while paying particular
attention to the impact of geopolitical risk. The explanatory variables
considered in the analysis, selected based on previous studies, are:

\begin{itemize}
\tightlist
\item
  The geopolitical risk (GPR) index of Caldara and Iacoviello (2022),
  which is constructed based on press articles covering 11 leading
  international newspapers, and defined as ``the kind of risk related to
  events such as wars, terrorist acts and political tensions, that can
  affect the normal and peaceful process of international relations''
  (Caldara and Iacoviello, 2022)\footnote{GPR indices have been used in
    various studies (see, Alqahatani et al., 2020; Ma et al., 2022;
    Mansour-Ichrakieh and Zeaiter, 2019; Wang et al., 2022; Wu et al.,
    2022).};
\item
  An interaction of GPR with the COVID-19 pandemic and Russian-Ukraine
  war (), where is a dummy variable taking the value of 1 during the
  COVID-19 outbreak and war period (January 02, 2020-- July 01, 2022)
  and 0 otherwise;
\item
  The US economic policy uncertainty (US EPU) index of Baker et
  al.~(2016), constructed based on US newspaper articles reflecting
  uncertainties in US economic policies;
\item
  The CBOE VIX index, which captures the 30-day expected volatility of
  the US stock market and is often used a proxy of fear among investors,
  not only in the US but across the global stock markets;
\item
  The log returns on the S\&P 500 Composite Index, which is used as a
  proxy for the performance of the global stock markets;
\item
  The US Treasury spread, computed as ``10-Year Treasury Constant
  Maturity Minus 2-Year Treasury Constant Maturity'', which reflects the
  shape of the US yield curve;
\item
  The TED spread, computed as the 3-month LIBOR USD rate minus the
  3-month US Treasury Bill rate, which captures short-term liquidity
  risk;
\item
  Default spread, computed as the yield on Moody's BAA-rated bonds minus
  the yield on AAA-rated corporate bonds, reflecting corporate credit
  conditions;
\item
  US business conditions, measured by the Aruoba-Diebold-Scotti (ADS)
  index of Aruoba et al.~(2009), which measures real business conditions
  on a daily basis;
\item
  The US inflation expectation, as measured by the 5-Year Breakeven
  Inflation Rate (T5YIE);
\end{itemize}

We report the estimated coefficients of Equation~\ref{eq-reg} in
Table~\ref{tbl-reg1} for return spillovers and in Table~\ref{tbl-reg2}
for volatility spillovers\footnote{We ensure that all variables entered
  in the regression are stationary, whether in their original levels or
  transformed (e.g.~change) levels.}

\begin{equation}\protect\hypertarget{eq-reg}{}{TOTAL_t=c+b_1GPR_{t-1}+b_2GPR_{t-1}.DCOVID+b_{it}X_{t-1}+e_t}\label{eq-reg}\end{equation}

where \(TOTAL_t\) is the total spillover index in the system of return
or volatility across A\&D companies, estimated at the lower, middle, or
higher quantiles; \(GPR\_{t-1}\) is the lagged value of the geopolitical
risk; \(GPR_{t-1}.DCOVID\) is the interaction term between GPR and the
COVID-19 and Russian-Ukraine war period; \(X_{t-1}\) is the vector of
the lagged value of control variables, described above, and is the
residual term. Except for GPR index, data on the other explanatory
variables are collected from Refinitiv DataStream.

\hypertarget{tbl-reg1}{}
\begin{table}[H]
\caption{\label{tbl-reg1}Drivers of return spillovers across A\&D companies for the full sample
period }\tabularnewline

\centering
\resizebox{\linewidth}{!}{
\begin{tabular}[t]{lllllll}
\toprule
\multicolumn{1}{c}{ } & \multicolumn{2}{c}{Middle quantile} & \multicolumn{2}{c}{Upper quantile} & \multicolumn{2}{c}{Lower quantile} \\
\cmidrule(l{3pt}r{3pt}){2-3} \cmidrule(l{3pt}r{3pt}){4-5} \cmidrule(l{3pt}r{3pt}){6-7}
Variable & Coefficient & Prob. & Coefficient & Prob. & Coefficient & Prob.\\
\midrule
GPRD(-1) & -0.033 & 0.000 & -0.003 & 0.000 & -0.003 & 0.000\\
GPRD(-1)*DCOVID & 0.069 & 0.000 & 0.007 & 0.000 & 0.006 & 0.000\\
USEPU(-1) & 0.008 & 0.004 & 0.001 & 0.040 & 0.001 & 0.048\\
VIX(-1) & 0.164 & 0.004 & 0.011 & 0.062 & 0.028 & 0.000\\
SP500(-1) & 24.392 & 0.027 & 0.075 & 0.957 & 6.462 & 0.000\\
\addlinespace
TERM SPREAD(-1) & -0.733 & 0.809 & 0.470 & 0.269 & -0.463 & 0.300\\
TED SPREAD (-1) & -0.079 & 0.000 & -0.005 & 0.002 & -0.003 & 0.122\\
DEFAULT SPREAD(-1) & 19.867 & 0.000 & 1.401 & 0.000 & 0.683 & 0.001\\
ADS BUS CONDITION INDEX(-1) & 0.713 & 0.000 & 0.071 & 0.000 & 0.066 & 0.000\\
US INFLATION(-1) & 2.478 & 0.001 & 0.122 & 0.134 & -0.474 & 0.000\\
\addlinespace
C & 42.044 & 0.000 & 91.401 & 0.000 & 92.879 & 0.000\\
 &  &  &  &  &  & \\
Adjusted R-squared & 0.640 & 8.408 & 0.392 & 0.867 & 0.387 & 0.901\\
F-statistic & 468.741 & 0.140 & 170.260 & 0.381 & 166.738 & 0.360\\
Prob(F-statistic) & 0.000 & 86.891 & 0.000 & 42.132 & 0.000 & 38.364\\
\bottomrule
\multicolumn{7}{l}{\textsuperscript{a} This table presents the estimated coefficients of the regression model in Equation (7)}\\
\multicolumn{7}{l}{based on a covariance estimator that accounts for the presence of heteroscedasticityand}\\
\multicolumn{7}{l}{autocorrelation (HAC). The sample period is 23 August 2010 –July 1, 2022.}\\
\end{tabular}}
\end{table}

\hypertarget{tbl-reg2}{}
\begin{table}[H]
\caption{\label{tbl-reg2}Drivers of volatility spillovers across A\&D companies for the full
sample period }\tabularnewline

\centering
\resizebox{\linewidth}{!}{
\begin{tabular}[t]{lllllll}
\toprule
\multicolumn{1}{c}{ } & \multicolumn{2}{c}{Middle quantile} & \multicolumn{2}{c}{Upper quantile} & \multicolumn{2}{c}{Lower quantile} \\
\cmidrule(l{3pt}r{3pt}){2-3} \cmidrule(l{3pt}r{3pt}){4-5} \cmidrule(l{3pt}r{3pt}){6-7}
Variable & Coefficient & Prob. & Coefficient & Prob. & Coefficient & Prob.\\
\midrule
GPRD(-1) & -0.061 & 0.000 & -0.002 & 0.040 & -0.042 & 0.000\\
GPRD(-1)*DCOVID & 0.115 & 0.000 & 0.001 & 0.194 & 0.072 & 0.000\\
USEPU(-1) & 0.035 & 0.000 & 0.000 & 0.257 & 0.014 & 0.000\\
VIX(-1) & 0.499 & 0.000 & 0.000 & 0.933 & 0.416 & 0.000\\
SP500(-1) & 84.600 & 0.000 & -1.099 & 0.419 & 61.075 & 0.000\\
\addlinespace
TERM SPREAD(-1) & -7.221 & 0.204 & -1.217 & 0.004 & -6.611 & 0.089\\
TED SPREAD (-1) & -0.019 & 0.482 & 0.000 & 0.923 & -0.055 & 0.003\\
CORPORATE CREDIT CONDITIONS(-1) & 14.191 & 0.000 & 0.060 & 0.695 & 16.357 & 0.000\\
ADS\_BUS\_CONDITION\_INDEX(-1) & 1.001 & 0.000 & 0.001 & 0.884 & 0.686 & 0.000\\
US INFLATION(-1) & 3.410 & 0.017 & -0.034 & 0.602 & 4.760 & 0.000\\
\addlinespace
C & 23.931 & 0.000 & 94.961 & 0.000 & 33.540 & 0.000\\
 &  &  &  &  &  & \\
Adjusted R-squared & 0.584 & 14.623 & 0.012 & 0.792 & 0.613 & 10.418\\
F-statistic & 369.986 & 0.163 & 4.247 & 0.858 & 416.551 & 0.131\\
Prob.(F-statistic) & 0.000 & 80.756 & 0.000 & 2.233 & 0.000 & 85.458\\
\bottomrule
\multicolumn{7}{l}{\textsuperscript{a} This table presents the estimated coefficients of the regression model in Equation (7) based}\\
\multicolumn{7}{l}{on a covariance estimator that accounts for the presence of heteroscedasticityand}\\
\multicolumn{7}{l}{autocorrelation (HAC). The sample period is 23 August 2010 –July 1, 2022.}\\
\end{tabular}}
\end{table}

Starting with the drivers of the TCI of returns, Table~\ref{tbl-reg1}
shows that many of the estimated coefficients of explanatory variables
are not necessarily the same across the middle and upper/lower quantile
spillovers. However, the GPR index is a significant driver of return
spillovers at all quantiles, and its effect is positive and significant
for all cases after controlling for the COVID-19 outbreak period, which
includes the Russian-Ukraine war sub-period. This suggests that the
heightened level of geopolitical risk around the war period has led to
increased return spillovers across A\&D stocks. Regarding the control
variables, we notice that S\&P500 returns, VIX, default spread, and
business conditions positively impact return spillovers, irrespective of
the quantile, bearing in mind that their magnitude is the largest at the
middle quantile. Table~\ref{tbl-reg2} considers the results on the
drivers of volatility spillovers. The results point to the exact impact
of GPR on volatility spillovers, especially when the interaction term is
considered, except at the upper quantile. Among the other explanatory
variables, we highlight the significant role played by corporate credit
conditions, stock market volatility, short-term liquidity risk, and real
business conditio

\hypertarget{conclusion}{%
\section{Conclusion}\label{conclusion}}

This study analyses the return and volatility connectedness of a sample
of major aerospace and defence stocks in normal and extreme market
periods using a quantile-based VAR approach of connectedness, which
captures the system of connectedness at lower, middle, and upper parts
of the conditional distribution of both returns and volatility. Then,
the drivers of the connectedness index estimated across various
quantiles are revealed, notably the geopolitical risk index. The main
results suggest evidence of variation in the quantile structure of the
system of connectedness among major aerospace and defence stocks. The
network topology analysis shows that shocks propagate more strongly at
both lower and upper tails of the conditional distribution than at the
conditional median, suggesting that the structure of spillovers at both
lower tails is dissimilar to that observed at the conditional median.
However, the magnitude of bilateral connections is smaller at the tails
relative to those at the median, but they are more apparent. In the
latter, connectedness is stronger within countries, but the volatility
and return systems are less connected overall. These results suggest
that the evolution of the dependence structure at the tails is notable
and should not be overlooked. In other words, the system-wide
connectedness of shocks in the A\&D industry can be masked when
connectedness measures are estimated at the conditional median or mean,
thus missing a significant portion of the spillover information related
to extraordinarily negative and positive shocks and extremely low and
high volatility states. Accordingly, applying quantile-based models of
connectedness is recommended as a natural extension to the pervasive
average-based models. This finding result matters for asset pricing and
allocation under various market conditions. The application of a
time-varying analysis shows that the degree of tail-dependence varies
with time and intensifies during periods of economic and
geopolitical-conflict turmoil. Lower-tail dependence is positively
correlated with upper-tail dependence, suggesting that extreme adverse
events are associated with an increase in stabilising lower-tail
connectedness coupled with a concurrent increase in destabilising
upper-tail connectedness. Intuitively, the main findings generally
concord with the literature on the return and volatility spillovers
under various market conditions in the global stock markets, which
reflect the nature of A\&D stocks despite their price outperformance
during periods of intensified geopolitical risk. Furthermore, the
relative tail dependence calculation indicates an asymmetry between the
behaviour of return (volatility) spillovers in lower and upper
quantiles. The findings on the extreme connectedness at upper and lower
tails offer a nuanced view of the importance of tail risk propagation
within the network system of aerospace and defence stocks, which point
to the utility of adopting quantile-based methods of spillovers to
differentiate between the networks of spillovers under various market
conditions. This might concern the supervision of risk under extreme
market conditions. In fact, by extending our knowledge regarding the
effects of the size and sign of the spillovers on the system of
connectedness among significant aerospace and defence stocks,
policymakers can use appropriate policy tools and surveillance
mechanisms to manage potential adversative impacts occurring from
extreme risk spillovers in the aerospace and defence industry.
Otherwise, focusing only on the average shocks within the system of
connectedness will likely lead to formulating and applying inappropriate
and insufficient stabilising policies during extreme events.

Additional analysis reveals the importance of geopolitical risk at the
end of the sample period in driving the spillovers of both returns and
volatility without underestimating the significant role played by some
macroeconomic and financial variables. Therefore, market participants
should closely examine the geopolitical risk levels for making
inferences on market integration in the aerospace and defence industry
and the diversification possibilities across various market conditions.

\hypertarget{appendix}{%
\section{Appendix}\label{appendix}}

\hypertarget{tbl-ref}{}
\begin{table}[H]
\caption{\label{tbl-ref}Table A1: Size information for our study's sample }\tabularnewline

\centering
\resizebox{\linewidth}{!}{
\begin{tabular}[t]{lrrr}
\toprule
Company Name & Market Capitalistation & Total Revenue 2022 & Total Revenue 2021\\
\midrule
Raytheon Technologies Corp & 147,447,678,880 & 64,388,000,000 & 67,074,000,000\\
Boeing Co & 122,950,199,509 & 62,286,000,000 & 66,608,000,000\\
Lockheed Martin Corp & 122,335,912,229 & 67,044,000,000 & 65,984,000,000\\
Airbus SE & 103,410,072,192 & 59,283,132,119 & 62,888,484,589\\
Northrop Grumman Corp & 72,539,645,185 & 35,667,000,000 & 36,602,000,000\\
\addlinespace
General Dynamics Corp & 64,084,906,210 & 38,469,000,000 & 39,407,000,000\\
Safran SA & 61,320,053,976 & 17,203,237,615 & 20,893,621,575\\
L3Harris Technologies Inc & 40,447,265,742 & 17,814,000,000 & 17,062,000,000\\
TransDigm Group Inc & 40,249,692,781 & 4,798,000,000 & 5,429,000,000\\
BAE Systems PLC & 33,378,358,649 & 26,357,384,087 & 26,410,065,616\\
\addlinespace
Thales SA & 30,091,644,275 & 18,772,594,040 & 18,407,111,839\\
HEICO Corp & 20,909,217,802 & 1,865,682,000 & 2,208,322,000\\
AECC Aviation Power Co Ltd & 17,702,958,771 & 4,388,141,416 & 5,368,648,696\\
Howmet Aerospace Inc & 17,352,985,409 & 4,972,000,000 & 5,663,000,000\\
Avic Shenyang Aircraft Co Ltd & 16,570,124,199 & 4,186,345,594 & 5,366,470,728\\
\addlinespace
Textron Inc & 15,054,696,965 & 12,382,000,000 & 12,869,000,000\\
Dassault Aviation SA & 14,937,859,878 & 6,706,878,359 & 8,237,497,442\\
MTU Aero Engines AG & 13,157,580,767 & 4,760,930,359 & 5,704,195,205\\
Rolls-Royce Holdings PLC & 11,078,227,000 & 15,711,609,719 & 15,176,892,376\\
Avic XiAn Aircraft Industry Group Co Ltd & 10,596,757,161 & 5,131,690,866 & 5,147,867,743\\
\addlinespace
Singapore Technologies Engineering Ltd & 8,369,671,163 & 5,419,248,997 & 5,702,642,698\\
\bottomrule
\multicolumn{4}{l}{\rule{0pt}{1em}\textit{Note: }}\\
\multicolumn{4}{l}{\rule{0pt}{1em}This data is source from Refinitiv Eikon and all values are in US Dollars.  The Market Capitalisation is a weight average of the 2022 daily values}\\
\end{tabular}}
\end{table}

\begin{figure}[H]

{\centering \includegraphics{defenceRR_files/figure-pdf/fig-mktcapshare-1.pdf}

}

\caption{\label{fig-mktcapshare}Treemap of market capitalisation of our
study's sample}

\end{figure}

\begin{figure}[H]

{\centering \includegraphics{defenceRR_files/figure-pdf/fig-prices-1.pdf}

}

\caption{\label{fig-prices}Figure A3: Price series levels}

\end{figure}

\begin{figure}[H]

{\centering \includegraphics{defenceRR_files/figure-pdf/fig-rtns-1.pdf}

}

\caption{\label{fig-rtns}Figure A3: Daily log returns}

\end{figure}

\begin{figure}[H]

{\centering \includegraphics{defenceRR_files/figure-pdf/fig-vols-1.pdf}

}

\caption{\label{fig-vols}Figure A4: Daily Volatilities}

\end{figure}

\hypertarget{tbl-sumrtn}{}
\begin{table}[H]
\caption{\label{tbl-sumrtn}Table A2: Summary statistics of daily returns }\tabularnewline

\centering
\resizebox{\linewidth}{!}{
\begin{tabular}[t]{lllllllll}
\toprule
A\&D Stock & Mean & Median & Std..Dev. & Skewness & Kurtosis & Jarque.Bera & ADF & PP\\
\midrule
RAYTHEON\_TECHNOLOGIES & 0.0003 & 0.0000 & 0.0157 & -0.3658 & 18.8916 & 32636.5*** & -21.2901*** & -56.4338***\\
LOCKHEED\_MARTIN & 0.0006 & 0.0005 & 0.0132 & -0.7847 & 18.2347 & 30248.2*** & -56.5766*** & -56.7477***\\
BOEING & 0.0003 & 0.0000 & 0.0226 & -0.5643 & 26.2666 & 69973.9*** & -18.1156*** & -52.3339***\\
AIRBUS & 0.0005 & 0.0002 & 0.0222 & -0.3905 & 16.9639 & 25224.5*** & -41.3354*** & -53.1875***\\
NORTHROP\_GRUMMAN & 0.0007 & 0.0005 & 0.0142 & -0.1678 & 10.8567 & 7974.9*** & -57.6092*** & -57.9145***\\
\addlinespace
GENERAL\_DYNAMICS & 0.0004 & 0.0003 & 0.0138 & -0.4154 & 9.2145 & 5069.4*** & -56.0303*** & -56.0601***\\
L3HARRIS\_TECHNOLOGIES & 0.0006 & 0.0004 & 0.0156 & -0.3236 & 13.3721 & 13927.4*** & -37.8971*** & -58.3311***\\
SAFRAN & 0.0005 & 0.0000 & 0.0208 & -0.5873 & 23.2326 & 52968.0 & -27.1859*** & -54.0423***\\
TRANSDIGM\_GROUP & 0.0007 & 0.0006 & 0.0204 & -0.8467 & 26.8661 & 73823.1 & -27.2219*** & -58.6233***\\
BAE\_SYSTEMS & 0.0003 & 0.0000 & 0.0145 & 0.0418 & 7.8111 & 2985.9 & -54.9012*** & -54.8972***\\
\addlinespace
THALES & 0.0005 & 0.0000 & 0.0157 & 0.3395 & 10.4513 & 7219.4 & -52.8929*** & -52.8478***\\
AECC\_AVIATION\_POWER\_\_A\_ & 0.0004 & 0.0000 & 0.0283 & -0.0139 & 6.3355 & 1434.8 & -50.3413*** & -50.2565***\\
HEICO & 0.0009 & 0.0004 & 0.0198 & 0.2280 & 11.1453 & 8582.7 & -37.9681*** & -57.4079***\\
AVIC\_SHENYANG\_AIRCRAFT\_\_A\_ & 0.0007 & 0.0000 & 0.0317 & -0.1037 & 5.3171 & 697.9 & -50.4755*** & -50.432***\\
TEXTRON & 0.0004 & 0.0000 & 0.0215 & -0.3143 & 13.2261 & 13536.4 & -56.7672*** & -56.7572***\\
\addlinespace
HOWMET\_AEROSPACE & 0.0002 & 0.0000 & 0.0250 & -0.3118 & 13.7557 & 14968.7 & -55.6534*** & -55.6534***\\
AVIC\_XI\_AN\_AIRCRAFT\_INDUSTRY\_GROUP\_\_A\_ & 0.0003 & 0.0000 & 0.0274 & -0.1139 & 6.1919 & 1320.5 & -51.6737*** & -51.6765***\\
DASSAULT\_AVIATION & 0.0003 & 0.0000 & 0.0177 & 0.2804 & 10.4996 & 7293.6 & -59.0881*** & -59.3442***\\
MTU\_AERO\_ENGINES\_\_XET\_\_HLDG\_ & 0.0004 & 0.0000 & 0.0196 & -0.2349 & 14.6874 & 17643.4 & -53.5665*** & -53.5378***\\
ROLLS\_ROYCE\_HOLDINGS & -0.0002 & 0.0000 & 0.0263 & 0.8073 & 25.6142 & 66286.0 & -42.2588*** & -53.3912***\\
\addlinespace
SINGAPORE\_TECHS\_ENGR\_ & 0.0001 & 0.0000 & 0.0121 & -0.2655 & 9.6055 & 5663.1 & -58.7849*** & -58.7819***\\
\bottomrule
\end{tabular}}
\end{table}

\hypertarget{tbl-sumvol}{}
\begin{table}[H]
\caption{\label{tbl-sumvol}Table A3: Summary statistics of daily volatilies }\tabularnewline

\centering
\resizebox{\linewidth}{!}{
\begin{tabular}[t]{lllllllll}
\toprule
A\&D Stock & Mean & Median & Std..Dev. & Skewness & Kurtosis & Jarque.Bera & ADF & PP\\
\midrule
RAYTHEON\_TECHNOLOGIES & 0.0002 & 0.0000 & 0.0010 & 14.3977 & 270.2230 & 9315605 & -8.8644*** & -71.0605***\\
LOCKHEED\_MARTIN & 0.0002 & 0.0000 & 0.0007 & 16.4325 & 350.1678 & 15682053 & -10.8166*** & -57.4850***\\
BOEING & 0.0005 & 0.0001 & 0.0026 & 16.0817 & 339.0628 & 14697726 & -9.4739*** & -59.9743***\\
AIRBUS & 0.0005 & 0.0001 & 0.0020 & 17.2260 & 424.2219 & 23033869 & -11.5572*** & -65.0529***\\
NORTHROP\_GRUMMAN & 0.0002 & 0.0000 & 0.0006 & 11.4526 & 195.7092 & 4856762 & -10.5620*** & -55.0415***\\
\addlinespace
GENERAL\_DYNAMICS & 0.0002 & 0.0000 & 0.0005 & 10.7424 & 180.7454 & 4133764 & -9.4617*** & -64.1503***\\
L3HARRIS\_TECHNOLOGIES & 0.0002 & 0.0001 & 0.0009 & 13.5332 & 273.2503 & 9512974 & -11.8003*** & -53.9115***\\
SAFRAN & 0.0004 & 0.0001 & 0.0020 & 19.3285 & 499.2195 & 31946613 & -11.6048*** & -62.0761***\\
TRANSDIGM\_GROUP & 0.0004 & 0.0001 & 0.0021 & 16.7444 & 377.0165 & 18184391 & -8.6389*** & -55.6789***\\
BAE\_SYSTEMS & 0.0002 & 0.0001 & 0.0006 & 9.3253 & 129.7845 & 2117775 & -16.0119*** & -50.1728***\\
\addlinespace
THALES & 0.0002 & 0.0001 & 0.0008 & 11.6051 & 190.9520 & 4625049 & -16.3795*** & -60.5916***\\
AECC\_AVIATION\_POWER\_\_A\_ & 0.0008 & 0.0001 & 0.0018 & 3.8458 & 18.4539 & 38428 & -9.8441*** & -62.2551***\\
HEICO & 0.0004 & 0.0001 & 0.0013 & 11.4285 & 201.3856 & 5142764 & -9.5040*** & -65.6277***\\
AVIC\_SHENYANG\_AIRCRAFT\_\_A\_ & 0.0010 & 0.0002 & 0.0021 & 3.2338 & 13.5866 & 19848 & -10.2645*** & -62.4013***\\
TEXTRON & 0.0005 & 0.0001 & 0.0016 & 10.1877 & 141.4459 & 2525317 & -10.6989*** & -55.9341***\\
\addlinespace
HOWMET\_AEROSPACE & 0.0006 & 0.0001 & 0.0022 & 13.6603 & 265.6162 & 8990159 & -15.2603*** & -62.3260***\\
AVIC\_XI\_AN\_AIRCRAFT\_INDUSTRY\_GROUP\_\_A\_ & 0.0007 & 0.0002 & 0.0017 & 4.0915 & 21.3110 & 51874 & -10.7828*** & -60.2592***\\
DASSAULT\_AVIATION & 0.0003 & 0.0001 & 0.0010 & 12.8240 & 286.0500 & 10416628 & -12.8417*** & -55.6529***\\
MTU\_AERO\_ENGINES\_\_XET\_\_HLDG\_ & 0.0004 & 0.0001 & 0.0014 & 11.7559 & 179.1791 & 4074035 & -9.3358*** & -66.7664***\\
ROLLS\_ROYCE\_HOLDINGS & 0.0007 & 0.0001 & 0.0034 & 21.7076 & 718.3667 & 66237434 & -5.3944*** & -63.2794***\\
\addlinespace
SINGAPORE\_TECHS\_ENGR\_ & 0.0001 & 0.0000 & 0.0004 & 13.5671 & 279.9223 & 9984241 & -12.0964*** & -67.2677***\\
\bottomrule
\end{tabular}}
\end{table}

\hypertarget{references}{%
\section*{References}\label{references}}
\addcontentsline{toc}{section}{References}

\hypertarget{refs}{}
\begin{CSLReferences}{1}{0}
\leavevmode\vadjust pre{\hypertarget{ref-Ando.2022}{}}%
Ando, Tomohiro, Matthew Greenwood-Nimmo, and Yongcheol Shin. 2022.
{``{Quantile Connectedness: Modeling Tail Behavior in the Topology of
Financial Networks}.''} \emph{Management Science} 68 (4): 2401--31.
\url{https://doi.org/10.1287/mnsc.2021.3984}.

\leavevmode\vadjust pre{\hypertarget{ref-Diebold.2014}{}}%
Diebold, Francis X., and Kamil Yılmaz. 2014. {``{On the network topology
of variance decompositions: Measuring the connectedness of financial
firms}.''} \emph{Journal of Econometrics} 182 (1): 119--34.
\url{https://doi.org/10.1016/j.jeconom.2014.04.012}.

\end{CSLReferences}



\end{document}
